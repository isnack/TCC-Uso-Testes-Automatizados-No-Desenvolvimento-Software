%\chapter*{Introdução}
\begin{flushleft}
	\vspace{1.2em}
	\textbf{\large INTRODUÇÃO}
	\vspace{2.9em}
\end{flushleft}
%\thispagestyle{empty}

\addcontentsline{toc}{chapter}{INTRODUÇÃO}
\stepcounter{chapter} %incrementa o número do capítulo

\par Com o constante avanço do desenvolvimento de \textit{software}, cada vez mais se encontram sistemas complexos e de alta criticidade espalhados pelo mundo. Quanto mais crítico e complexo é o sistema,  maior é a necessidade de que o sistema seja executado com precisão, minimizando os erros.

\par As empresas procuram aprimorar e aperfeiçoar o desenvolvimento de \textit{softwares}, fazendo com que os usuários tenham mais confiança e segurança na utilização dos sistemas.

\par Devido à complexidade e inúmeras dificuldades relacionadas ao processo de desenvolvimento dos \textit{softwares}, que envolvem as questões humanas, burocráticas e políticas. As empresas têm um grande desafio para controlar a qualidade destes \textit{softwares} para os usuários finais. Os sistemas precisam fazer corretamente o que foi solicitado pelo cliente, de forma segura, eficiente e de fácil manutenção e evolução \cite{ime},s.p.

\par Para auxiliar as empresas de desenvolvimento de \textit{software}, surgiram processos de testes automatizados. Segundo \citeonline{qualister}, "o propósito da automação de testes pode ser resumidamente descrito como a aplicação de estratégias e ferramentas, tendo em vista a redução do envolvimento humano em atividades manuais repetitivas".

\par Há 10 anos foi realizada uma pesquisa pela Forrester Research Inc em que 9\% das empresas(Estados Unidos e Reino Unido) utilizavam testes em seus processos, a fim de diminuir correções e retrabalhos nos \textit{softwares}, aumentando a satisfação do usuário final com um sistema mais preciso possível \cite{qualister}.

\par Diariamente, algumas tarefas são repetidas diversas vezes, ocasionando um desgaste natural em seus executores. Tarefas repetitivas são, muitas vezes, morosas àqueles que as executam, o que pode gerar falhas e comprometer o procedimento a ser executado. No mundo de desenvolvimento de \textit{softwares} não é diferente. Há tarefas repetitivas e custosas aos que as colocam em execução. Muitas vezes, pessoas erram e podem ocasionar falhas nestes procedimentos. Certos problemas ocasionados podem gerar desconforto para as empresas, como a insatisfação do cliente final com determinados problemas e erros que não foram verificados com cautela.

\par Os testes automatizados surgiram para auxiliar este cenário acima citado. São eles os responsáveis por verificar constantemente o bom funcionamento de uma aplicação, a fim de verificar se há problemas que possam comprometer o sistema. Com isso, um \textit{software} com maior qualidade é disponibilizado ao cliente final, pois os erros comuns são encontrados através desta abordagem. Além do mais, os testes automatizados se tornam eficazes, pois é possível criar cenários complexos e elaborados para testes com o intuito de analisar toda e qualquer probabilidade de erro, o que toma muito tempo hábil de um desenvolvedor caso o mesmo fosse feito manualmente.

\par Os testes podem ser resumidamente descritos como a aplicação de estratégias e ferramentas tendo em vista a redução do envolvimento humano em atividades manuais repetitivas.

\par Para isso, usa-se o teste de \textit{software} para poder testar a aplicação no desenvolvimento, o qual pode incluir uma aplicação \textit{web}, aplicativos para celular, entre outros. Nesse procedimento é possível fazer verificações do funcionamento das aplicações para que se possa evitar erros futuros. Este uso representa ganhos para a empresa, tais como: significativa economia de esforço, aumento da qualidade do sistema, diminuição do custo de manutenção e o aumento da satisfação do cliente, dentre outros.

\par Ao fazer o teste de \textit{software}, entrega-se um produto final de alta qualidade e poucos erros. A empresa tem redução de gastos com a manutenção de aplicações que possam apresentar erros. Por isso, é tão importante falarmos em qualidade, pois está diretamente relacionada com a satisfação. Segundo \citeonline{teste}, "a qualidade não é apenas um diferencial e sim uma meta para se tornar uma empresa mundialmente conhecida."
\par O objetivo geral desse trabalho é demonstrar o funcionamento dos testes automatizados, e para que o objetivo geral fosse alcançado, ele foi dividido em alguns específicos.


\begin{itemize}
    \item Discussão dos testes automatizados;
    \item Demonstração das ferramentas;
    \item Demonstração da utilização dos testes automatizados;
    \item Integração contínua.
\end{itemize}

\par Serão descritas técnicas, tipos de testes e análise de ferramentas, objetivando a análise de processo de testes de \textit{software} automatizado como um processo de desenvolvimento.

\par O trabalho é composto de quatro capítulos, sendo o atual uma breve introdução sobre o tema proposto e seus objetivos. No segundo capítulo foram descritas as teorias sobre as tecnologias utilizadas para o desenvolvimento do projeto. O terceiro, relatou as metodologias usadas na pesquisa, tais como: tipo de pesquisa, contexto de pesquisa, instrumentos e procedimentos.
No quarto capítulo foi realizada a discussão de resultados ressaltando o que se obteve com a realização do trabalho. Por fim, a conclusão.